\begin{figure}[h]
  \centering
  \hspace*{-0.75cm}
  \begin{tikzpicture}
    \node (x) at (0,0) {\small$[\vb{p}_0, \vb{p}_t,\vb{w}]$};
 
    \node[conv,rotate=90,minimum width=4.5cm] (l0) at (2,0) 
      {\small Linear + ReLU};

    \node[conv,rotate=90,minimum width=4.5cm] (l1) at (4.0, 0)
      {\small Linear + ReLU};

    \node[conv,rotate=90,minimum width=4.5cm] (l2) at (6.0, 0)
      {\small Linear + ReLU};

    \node[conv,rotate=90,minimum width=4.5cm] (l3) at (8.0, 0)
      {\small Linear + ReLU};

    \node[conv,rotate=90,minimum width=4.5cm] (l4) at (10.0, 0)
      {\small Linear + ReLU};

    \node[conv,rotate=90,minimum width=4.5cm] (l5) at (12.0, 0)
      {\small Linear + ReLU};
    
      \node (f) at (13.5,0) {\small$\vb{f}$};
    
      \draw[->] (x) -- (l0);
    \draw[->] (l0) -- (l1) node[pos=0.5,sloped,above] {$512$};
    \draw[->] (l1) -- (l2) node[pos=0.5,sloped,above] {$256$};
    \draw[->] (l2) -- (l3) node[pos=0.5,sloped,above] {$128$};
    \draw[->] (l3) -- (l4) node[pos=0.5,sloped,above] {$64$};
    \draw[->] (l4) -- (l5) node[pos=0.5,sloped,above] {$32$};
    \draw[->] (l5) -- (f) node[pos=0.5,sloped,above] {$16$};
    
  \end{tikzpicture}
  \vskip 6px
  \caption{The CFE NN transforms the input vector of size $2\cdot2(O+U+T)+N$
  into a force vector $\vb{f}$ that can be added to the $\vb{w}$ coefficients as
external force. The output size of each layer matches the input size of the
following layer, and a ReLU non-linearity is applied after each layer.}
  \label{fig:nn-architecture}
\end{figure}
